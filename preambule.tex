% !TEX root =  presentation.tex
\usepackage{times}
\usepackage{lmodern}
\usepackage{multirow}
\usepackage[T1]{fontenc}

\usepackage{amsmath,amsthm,mathrsfs,amssymb,amsfonts,dsfont,nicefrac,stmaryrd,yhmath}
\usepackage{mathtools,cancel,multicol}
\mathtoolsset{showonlyrefs=true}

\usepackage{csquotes} 
\usepackage{layout}
\usepackage{cite} % better citations
\usepackage{subcaption}
%\usepackage[belowskip=-20pt]{caption}
\usepackage{tikz,pgfplots}
\pgfplotsset{compat=newest}
\usepackage{pgfmath,pgffor}
\usetikzlibrary{plotmarks}
\usepackage{pgfplotstable}
\usetikzlibrary{arrows,positioning,shapes} 
\usepackage{animate}
\usepackage{subcaption}
\usepackage{pgfpages}
\usepackage{changepage}
\usepackage{xstring}
\usepackage{soul}
\usepgfplotslibrary{statistics}


\newsavebox\foobox
\newcommand\slbox[2]{%
	\FPdiv{\result}{#1}{57.296}% CONVERT deg TO rad
	\FPtan{\result}{\result}%
	\slantbox[\result]{#2}%
}%
\newcommand{\slantbox}[2][30]{%
	\scalebox{1}[.7]{\mbox{%
			\sbox{\foobox}{#2}%
			\hskip\wd\foobox
			\pdfsave
			\pdfsetmatrix{1 0 #1 1}%
			\llap{\usebox{\foobox}}%
			\pdfrestore
}}}
\newcommand\rotslant[3]{\rotatebox{#1}{\slbox{#2}{#3}}}

\usetikzlibrary{calc,intersections,through,backgrounds,arrows,positioning,shapes.geometric}

\tikzstyle{startstop} = [rectangle, rounded corners, very thick, minimum width=3cm, minimum height=1cm,text centered, draw=black]%fill=red!30
\tikzstyle{io} = [trapezium, trapezium left angle=70, trapezium right angle=110, minimum width=3cm, minimum height=1cm, text centered, draw=black, fill=blue!30]
\tikzstyle{decision} = [diamond, very thick, minimum width=1cm, minimum height=1cm, text centered, text width=2.5cm, draw=green]%fill=green!30]
\tikzstyle{process} = [rectangle, very thick, rounded corners, minimum width=3cm, minimum height=1cm, text centered, draw=blue] %fill=orange!30]
\tikzstyle{arrow} = [thick, ->, >=stealth]


\tikzset{
    %Define standard arrow tip
    >=stealth',
    %Define style for boxes
    punkt/.style={
           rectangle,
           rounded corners,
           draw=black, very thick,
           text width=6.5em,
           minimum height=2em,
           text centered},
    DAT/.style={
    rectangle,
           %shape aspect=3,diamond,
           %rounded corners,
           draw=black, very thick,
           text width=6.5em,
           minimum height=2em,
           text centered},
    % Define arrow style
    pil/.style={
           ->,
           thick,
           shorten <=2pt,
           shorten >=2pt},
%%           
    punkt2/.style={
    rectangle,
           rounded corners, draw=black, fill=color3,very thick,  text width=5.5em,  text centered},
%
    diam/.style={
    diamond,
           rounded corners, draw=black, fill=color1,very thick,  text centered},
%
    tria/.style={
    rounded rectangle,
           rounded corners, draw=black, fill=color2,very thick, text width=6.2em,  text centered},
%            
    clo/.style={
    ellipse,
           rounded corners, draw=black, fill=color5 ,thick,  text centered},          
%
% \tikzstyle{cloud}    = [draw=red, thick, ellipse,fill=red!20, minimum height=4em];
    DAT2/.style={
    ellipse,
           rounded corners, draw=black, fill=color4 ,thick,  text centered}, 
%
    % Define arrow style
    pil/.style={
     ->,
           thick, shorten <=2pt, shorten >=2pt,}
}

\definecolor{blocky}{rgb}{0.902,0.945,0.976} 
\newcommand{\tcblo}{\textcolor{blocky}}

\newcommand{\tcr}{\textcolor{red}}
\newcommand{\tcn}{\textcolor{black}}
\newcommand{\tcw}{\textcolor{white}}

\newcommand{\sig}{\sigma}
\newcommand{\bssig}{\boldsymbol{\sigma}}

\newcommand{\m}{\boldsymbol{m}}

\newcommand{\R}{\mathbb{R}}
\newcommand{\C}{\mathbb{C}}
\newcommand{\Q}{\mathbb{Q}}
\newcommand{\N}{\mathbb{N}}
\newcommand{\Z}{\mathbb{Z}}
\newcommand{\K}{\mathbb{K}}
\newcommand{\Y}{\mathbb{Y}}

\renewcommand{\H}{\mathbb{H}}
\newcommand{\V}{\mathbb{V}}

\newcommand{\calB}{\mathcal{B}}
\newcommand{\calD}{\mathcal{D}}
\newcommand{\calC}{\mathcal{C}}
\newcommand{\calP}{\mathcal{P}}
\newcommand{\calS}{\mathcal{S}}
\newcommand{\calL}{\mathcal{L}}
\newcommand{\calT}{\mathcal{T}}
\newcommand{\calR}{\mathcal{R}}
\newcommand{\calU}{\mathcal{U}}
\newcommand{\calF}{\mathcal{F}}
\newcommand{\calE}{\mathcal{E}}
\newcommand{\calI}{\mathcal{I}}
\newcommand{\calO}{\mathcal{O}}
\newcommand{\lvl}{\calL}


\newcommand{\us}{u_{\bssig}}
\newcommand{\usi}{u_{\bssig_i}}
\newcommand{\um}{u_{\m}}
\newcommand{\umi}{u_{\m_i}}
\newcommand{\vm}{v_{\m}}
\newcommand{\vmi}{v_{\m_i}}
\newcommand{\vs}{v_{\bssig}}
\newcommand{\ws}{w_{\bssig}}
\newcommand{\ls}{l_{\bssig}}
\newcommand{\lsi}{l_{\bssig_i}}
\newcommand{\tl}{\tilde{l}}


\newcommand{\dis}{\displaystyle}
\newcommand{\abs}[1]{\left|#1\right|}
\newcommand{\eps}{\varepsilon}
\newcommand{\norme}[1]{\left\|#1\right\|}
\newcommand{\norm}[1]{\left\|#1\right\|}

\renewcommand{\leq}{\leqslant}
\renewcommand{\geq}{\geqslant}
\renewcommand{\tilde}{\widetilde}
\newcommand{\grad}{\nabla}
\newcommand{\scalaire}[2]{\left<#1\,,#2\right>}
\newcommand{\scalar}[2]{\left<#1\,,#2\right>}

\DeclareMathOperator{\atan}{atan}
\DeclareMathOperator{\atanB}{atan2}

\setbeamertemplate{items}[ball]
\setbeamertemplate{blocks}[rounded][shadow=true]
\setbeamertemplate{navigation symbols}{}
\usefonttheme{structurebold}

\defbeamertemplate{footline}{centered page number}
{%
  \hspace*{\fill}%
  \usebeamercolor[fg]{page number in head/foot}%
  \usebeamerfont{page number in head/foot}%
 { \hspace{\fill}}
 \insertpagenumber 
  {\hspace{\fill}
  \hspace*{\fill}\vskip2pt}%
}
\setbeamertemplate{footline}[centered page number]

\defbeamertemplate{footline}{centered frame number}
{%
  \hspace*{\fill}%
  \usebeamercolor[fg]{page frame in head/foot}%
  \usebeamerfont{page frame in head/foot}%
 { \hspace{\fill}}
 \insertframenumber 
  {\hspace{\fill}
  \hspace*{\fill}\vskip2pt}%
}
\setbeamertemplate{footline}[centered frame number]{}
%
\newcommand\tikzmark[1]{
  \tikz[remember picture,overlay] \coordinate (#1);
}

\usetikzlibrary{datavisualization, patterns}
\usepgfplotslibrary{fillbetween}
\usetikzlibrary{decorations.markings,intersections,calc}
\usetikzlibrary{decorations.pathreplacing}
\usetikzlibrary{fit}
\makeatletter
\tikzset{
  fitting node/.style={
    inner sep=0pt,
    fill=none,
    draw=none,
    reset transform,
    fit={(\pgf@pathminx,\pgf@pathminy) (\pgf@pathmaxx,\pgf@pathmaxy)}
  },
  reset transform/.code={\pgftransformreset}
}

\makeatother

\usepgfplotslibrary{colormaps}
\usepgfplotslibrary{colorbrewer}
%\setbeameroption{show notes}
 %  \setbeameroption{show notes on second screen} %\setbeameroption{show notes on second screen=<location> (top, bottom, left)}
	%\setbeameroption{show only notes}
\usepgfplotslibrary{external}
%\tikzexternalize[mode= list and make, prefix=External_TikZ/]
%\tikzsetfigurename{Figure-} % change the name of the figures
%\tikzset{external/remake next} % force the next tikz to be remaked
%\tikzset{external/force remake} % force the all the next tikz to be remaked

\usepackage{tikz-3dplot}
\usepgfplotslibrary{groupplots}
\usetikzlibrary{overlay-beamer-styles}
\usetikzlibrary{spy}

\newlength{\plotwidth}
\newlength{\plotheight}
%\setlength{\plotheight}{7cm}
\setlength{\plotheight}{0.85\textheight}
%\setlength{\plotwidth}{10cm}
\setlength{\plotwidth}{0.9\textwidth}
\newlength{\plotwidthmesh}
\setlength{\plotwidthmesh}{0.49\textwidth}
%\newlength{\plotwidthmeshacoustic}
%\setlength{\plotwidthmeshacoustic}{0.22\textwidth}
\newcommand{\subplotwidth}{0.45\textwidth}
\newcommand{\subplotwidthhp}{0.49\textwidth}
\newcommand{\subplotheight}{\plotwidth}

\newcommand{\figpath}{Figures}
\newcommand{\FigurePath}{Figures}
\newcommand{\DataPath}{}