% !TEX root = presentation.tex

% Font and encoding packages
\usepackage{times}
\usepackage{lmodern}
\usepackage[T1]{fontenc}

% Math packages
\usepackage{amsmath, amsthm, mathrsfs, amssymb, amsfonts, dsfont, nicefrac, stmaryrd, yhmath}
\usepackage{mathtools}
\mathtoolsset{showonlyrefs=true}
\usepackage{cancel}
\usepackage{multirow}

% Language and citation packages
\usepackage{csquotes}
\usepackage{cite}

% Caption and subcaption packages
\usepackage{subcaption}

% Tikz and pgfplots packages
\usepackage{tikz}
\usetikzlibrary{plotmarks, arrows, positioning, shapes, calc, intersections, through, backgrounds}
\usepackage{pgfplots}
\pgfplotsset{compat=newest}
\usepackage{pgfmath, pgffor}
\usepgfplotslibrary{statistics, colorbrewer, groupplots}
\usepackage{pgfplotstable}
\usepackage{booktabs}
\usepackage{longtable}
\pgfplotstableset{
  begin table=\begin{longtable},
  end table=\end{longtable},
}

% Animation and page layout packages
\usepackage{animate}
\usepackage{pgfpages}
\usepackage{changepage}

% String and text manipulation packages
\usepackage{xstring}
\usepackage{soul}

% Define colors
\definecolor{blocky}{rgb}{0.902,0.945,0.976}

% Define Tikz styles
\tikzstyle{startstop} = [rectangle, rounded corners, very thick, minimum width=3cm, minimum height=1cm, text centered, draw=black]
\tikzstyle{io} = [trapezium, trapezium left angle=70, trapezium right angle=110, minimum width=3cm, minimum height=1cm, text centered, draw=black, fill=blue!30]
\tikzstyle{decision} = [diamond, very thick, minimum width=1cm, minimum height=1cm, text centered, text width=2.5cm, draw=green]
\tikzstyle{process} = [rectangle, very thick, rounded corners, minimum width=3cm, minimum height=1cm, text centered, draw=blue]
\tikzstyle{arrow} = [thick, ->, >=stealth]

% Define additional Tikz styles
\tikzset{
    >=stealth',
    punkt/.style={
           rectangle,
           rounded corners,
           draw=black,
           very thick,
           text width=6.5em,
           minimum height=2em,
           text centered
    },
    DAT/.style={
        rectangle,
        draw=black,
        very thick,
        text width=6.5em,
        minimum height=2em,
        text centered
    },
    pil/.style={
        ->,
        thick,
        shorten <=2pt,
        shorten >=2pt
    },
    punkt2/.style={
        rectangle,
        rounded corners,
        draw=black,
        fill=color3,
        very thick,
        text width=5.5em,
        text centered
    },
    diam/.style={
        diamond,
        rounded corners,
        draw=black,
        fill=color1,
        very thick,
        text centered
    },
    tria/.style={
        rounded rectangle,
        rounded corners,
        draw=black,
        fill=color2,
        very thick,
        text width=6.2em,
        text centered
    },
    clo/.style={
        ellipse,
        rounded corners,
        draw=black,
        fill=color5,
        thick,
        text centered
    },
    DAT2/.style={
        ellipse,
        rounded corners,
        draw=black,
        fill=color4,
        thick,
        text centered
    }
}

% Custom commands
\newcommand{\tcblo}{\textcolor{blocky}}
\newcommand{\tcr}{\textcolor{red}}
\newcommand{\tcn}{\textcolor{black}}
\newcommand{\tcw}{\textcolor{white}}
\newcommand{\sig}{\sigma}
\newcommand{\bssig}{\boldsymbol{\sigma}}
\newcommand{\m}{\boldsymbol{m}}
\newcommand{\R}{\mathbb{R}}
\newcommand{\C}{\mathbb{C}}
\newcommand{\Q}{\mathbb{Q}}
\newcommand{\N}{\mathbb{N}}
\newcommand{\Z}{\mathbb{Z}}
\newcommand{\K}{\mathbb{K}}
\newcommand{\Y}{\mathbb{Y}}
\renewcommand{\H}{\mathbb{H}}
\newcommand{\V}{\mathbb{V}}
\newcommand{\calB}{\mathcal{B}}
\newcommand{\calD}{\mathcal{D}}
\newcommand{\calC}{\mathcal{C}}
\newcommand{\calP}{\mathcal{P}}
\newcommand{\calS}{\mathcal{S}}
\newcommand{\calL}{\mathcal{L}}
\newcommand{\calT}{\mathcal{T}}
\newcommand{\calR}{\mathcal{R}}
\newcommand{\calU}{\mathcal{U}}
\newcommand{\calF}{\mathcal{F}}
\newcommand{\calE}{\mathcal{E}}
\newcommand{\calI}{\mathcal{I}}
\newcommand{\calO}{\mathcal{O}}
\newcommand{\lvl}{\calL}
\newcommand{\us}{u_{\bssig}}
\newcommand{\usi}{u_{\bssig_i}}
\newcommand{\um}{u_{\m}}
\newcommand{\umi}{u_{\m_i}}
\newcommand{\vm}{v_{\m}}
\newcommand{\vmi}{v_{\m_i}}
\newcommand{\vs}{v_{\bssig}}
\newcommand{\ws}{w_{\bssig}}
\newcommand{\ls}{l_{\bssig}}
\newcommand{\lsi}{l_{\bssig_i}}
\newcommand{\tl}{\tilde{l}}
\newcommand{\dis}{\displaystyle}
\newcommand{\abs}[1]{\left|#1\right|}
\newcommand{\eps}{\varepsilon}
\newcommand{\norme}[1]{\left\|#1\right\|}
\newcommand{\norm}[1]{\left\|#1\right\|}
\renewcommand{\leq}{\leqslant}
\renewcommand{\geq}{\geqslant}
\renewcommand{\tilde}{\widetilde}
\newcommand{\grad}{\nabla}
\newcommand{\scalaire}[2]{\left<#1\,,#2\right>}
\newcommand{\scalar}[2]{\left<#1\,,#2\right>}
\DeclareMathOperator{\atan}{atan}
\DeclareMathOperator{\atanB}{atan2}
\DeclareMathOperator*{\argmin}{arg\,min}